\documentclass[
  letterpaper,
  12pt,
  headsepline,
  pointlessnumbers,
  tablecaptionabove,
  headinclude,
  appendixprefix,
  idxtotoc,
  bibtotoc
]{scrartcl}

% Fonts
\usepackage{mathptmx} % Times font in main text and mathematical formulas
\usepackage{wasysym}  % Contains several special symbols
\usepackage{pifont}   % Enables Zapf dingbats (special symbols)
\usepackage[scaled=.95]{helvet} % Sans serif font to be used

% Language definition
\usepackage[english]{babel}
\usepackage{varioref} % better references with \vref

% Figures
\usepackage{graphicx}
\usepackage{sidecap}

% Some color in the document
\usepackage{color}

% Source code and pseudo code
\usepackage{listings}    % Source code listing support
\usepackage{algorithm}
\usepackage{algorithmic} % Algorithmen im Pseudo-Code,

\usepackage[unicode]{hyperref}
\usepackage[htt]{hyphenat}

% Style definitions

\selectlanguage{english}

% some nice colors
\definecolor{royalblue}{cmyk}{.93, .79, 0, 0}
\definecolor{lightblue}{cmyk}{.10, .017, 0, 0}
\definecolor{darkgreen}{rgb}{0,.7,0}
\definecolor{darkred}{rgb}{.7,0,0}
\definecolor{lightgray}{gray}{0.97}

\hypersetup{
              bookmarks={true},
              bookmarksopen={true},
              bookmarksopenlevel={0},
              bookmarksnumbered={true},
              breaklinks={true},
              colorlinks={false},
              pdfpagemode={UseOutlines},
              pdftitle={A short description of the main differences between JSBML and LibSBML},
              pdfauthor={Andreas Dr\"ager},
              pdfsubject={Software guide},
              pdfkeywords={LibSBML, JSBML, Java, SBML, API, LaTeX},
              pdfview={FitV},
              pdftex,
              pdffitwindow={true},
              pdfstartview={FitV},
              pdfnewwindow={false},
              pdfdisplaydoctitle={true},
              pdfhighlight={/P},
              plainpages={false},
              unicode={true},
              urlcolor={blue}
}

\lstset{language=Java,
captionpos=b,
basicstyle=\footnotesize\ttfamily\bfseries,
stringstyle=\color{darkred}\footnotesize\ttfamily,
keywordstyle=\color{royalblue}\bfseries\ttfamily,
ndkeywordstyle=\color{darkgreen},
numbers=left,
numberstyle=\footnotesize,
% backgroundcolor=\color{lightgray},
breaklines=true,
tabsize=2,
frame=single,
breakatwhitespace=true
% framexleftmargin=5mm,
% rulesepcolor=\color{lightgray}
% frameround=ttff
}
\pagestyle{headings}

\title{A short description of the main differences between JSBML and LibSBML}
\author{Andreas Dr\"ager}
\date{\today}

\begin{document}

\maketitle

\begin{abstract}
Although the libraries JSBML and LibSBML for working with files and data structures 
defined in the standard SBML (Systems Biology Markup Language) are very similar and 
share a common scope, users should be informed about their major differences to 
switch from one library from the other one more easily. To this end, the document at
hand gives a brief overview of the main differences between the Java\texttrademark{} 
application programming interfaces of both libraries.
\end{abstract}

\section{An extended type hierarchy}

Whenever multiple elements defined in the SBML specification share some attributes,
JSBML provides a common super class or at least a common interface that gathers methods
for manipulation of the shared properties. In this way, the type hierarchy of JSBML has
become more complex (see Fig.~\vref{fig:TypeHierarchy}). 
\begin{figure}[btp]
\centering
\includegraphics[width=\textwidth]{FullTypeHierarchy}
\caption[The type hierarchy in JSBML]{The type hierarchy of the main SBML constructs in JSBML}
\label{fig:TypeHierarchy}
\end{figure}
Just like in LibSBML, all elements extend the abstract type \verb!SBase!, but in JSBML, 
\verb!SBase! has become an interface. This allows more complex relations between derived 
data types. In contrast to LibSBML, \verb!SBase! in JSBML extends three other interfaces: 
\verb!Cloneable!, \verb!Serializable!, and \verb!TreeNode!. As all elements defined in JSBML 
override the \verb!clone()! method from the class \verb!java.lang.Object!, all JSBML 
elements can be deepely copied and are therefore ``cloneable". By extending the interface
\verb!Serializable!, it is possible to store JSBML elements in binary form without explicitly 
writing it to an SBML file. In this way, programs can easily load and save their in-memory
objects or send complex data structures through a network connection without the need of
additional file encoding and subsequent parsing. 
The third interface, \verb!TreeNode! is actually defined in Java's \verb!swing! package, but
defines a data type independent of any graphical information. It basically defines recursive
methods on hierarchically structured data types, such as iteration over all of its successors.
In this way, all instances of JSBML's \verb!SBase! interface can be directly passed to the
\verb!swing! class \verb!JTree! and hence be easily visualized. Listing~\vref{lst:Visualization}
demonstrates in a simple code example how to parse an SBML file and to immediately display
its content on a \verb!JFrame!.
\lstinputlisting[language=Java,float,caption={Parsing and visualizing the content of an SBML 
file},label=lst:Visualization]{posters/2010_ICSB_and_COMBINE/JSBMLvisualizer.java}
Fig.~\vref{fig:Visualization} shows an example output when applying the program from 
Listing~\vref{lst:Visualization} to SBML test model \texttt{case00026}.
\begin{SCfigure}[][t]
\includegraphics[width=.35\textwidth]{posters/2010_ICSB_and_COMBINE/JSBMLvisualizerTransparent}
\caption[Tree representatation of an SBML file]{A tree representation of the content of SBML test model \texttt{case00026}}
\label{fig:Visualization}
\end{SCfigure}
The \verb!ASTNode! class in JSBML also implements all these three interfaces and can hence
be cloned, serialized, and visualized in the same way.


\section{Abstract syntax trees}

Both libraries define a class \verb!ASTNode! for in-memory manipulation and evaluation
of abstract syntax trees that represent mathematical formulae and equations. These
can either pe parsed from a representation in \verb!C! language-like \verb!String!s,
or from a MathML representation. The JSBML \verb!ASTNode! provides various methods
to transform these trees to other formats, for instance, \LaTeX{} \verb!String!s.
In JSBML, several static methods allow easy creation of new syntax trees, for instance,
the following code
\begin{verbatim}
ASTNode myNode = ASTNode.plus(myLeftAstNode, myRightASTNode);
\end{verbatim}
creates a new instance of \verb!ASTNode! which represents the sum of the two other
\verb!ASTNode!s. In this way, even complex trees can be easily manipulated.

\section{The \texttt{ASTNodeCompiler} class}

This interface allows users to create customized interpreters for the
content of mathematical equations encoded in abstract syntax trees. It
is directly and recursively called from the \verb!ASTNode! class and returns
an \verb!ASTNodeValue! object, which wraps the possible evaluation results of
the interpretation. JSBML already provides several implementations of
this interface, for instance, \verb!ASTNode! objects can be directly translated
to LaTeX or MathML for further processing.

\section{Deprecation}

The intension of JSBML is to provide a Java library for the latest 
specification of SBML. Hence, JSBML provides methods and classes to
cover earlier releases of SBML as well, but these are often marked
as being deprecated to avoid creating models that refer to these 
elements.

\section{Exceptions}

Generally, JSBML throws more exceptions than LibSBML. This behavior helps
programmers and users to avoid creating invalid SBML data structures already
when dealing with these in memory. Examples are the \verb!ParseException! that
may be thrown if a given formula cannot be parsed properly into an \verb!ASTNode!
data structure, or \verb!InvalidArgumentException!s if inappropriate values are
passed to methods. For instance, an object representing a constant such as a
\verb!Parameter! whose constant attribute has been set to \verb!true! cannot be
used as the \verb!Variable! element in an \verb!Assignment!. Another example is
the \verb!InvalidArgumentException! that is thrown when trying to set an invalid
identifier \verb!String! for an instance of \verb!AbstractNamedSBase!. Hence,
you have to be aware of potential exceptions and errors when using JSBML, on the
other hand will this behavior prevent you from doing obvious mistakes.

\section{Initial assignments}

JSBML unifies all those elements that assign values to some other 
\verb!SBase! in SBML under the interface Assignment. This interface uses
the term Variable for the element whose value is to be changed depending
on some mathematical expression that is also present in the Assignment
(because Assignment extends the interface MathContainer). Therefore,
an Assignment contains methods such as \verb!set!-/\verb!getVariable(Variable v)!
and also \verb!isSetVariable()! and \verb!unsetVariable()!. In addition to that
JSBML also provides the method \verb!set!-/\verb!getSymbol(String symbol)! in the
InitialAssignment class to make sure that switching from LibSBML to
JSBML is quite smoothly. However, the preferred way in JSBML is to
apply the methods setVariable either with String or Variable instances
as arguments.

\section{Model history}

In earlier versions of SBML only the model itself could be associated with a
history, i.e., a description about the person(s) who build this model, including
names, e-mail addresses, modification and creation dates. Nowadays, it has
become possible to annotate each individual construct of an SBML model with such
a history. This is reflected by naming the corresponding object \verb!History!
in JSBML, whereas it is still called \verb!ModelHistory! in LibSBML. Hence, all
instances of \verb!SBase! in JSBML contain methods do access and manipulate its
\verb!History!. Furthermore, you will not find the classes \verb!ModelCreator!
and \verb!ModelCreatorList! because JSBML gathers its \verb!Creator! objects
in a generic \verb!List<Creator>! in the \verb!History!.

\section{The \texttt{MathContainer} interface}

This interface gathers all those elements that may contain mathematical
expressions encoded in abstract syntax trees (instances of \verb!ASTNode!).
The abstract class \verb!AbstractMathContainer! serves as actual super class
for most of the derived types.
 

\section{The classes \texttt{libSBML} and \texttt{JSBML}}

There is no class LibSBML because this library is called JSBML. You
can therefore only find a class JSBML. This class provides similar
methods as the LibSBML class in LibSBML, such as methods to
\begin{itemize}
 \item parse a formula \verb!String! into a corresponding \verb!ASTNode! object
 \item obtain a \verb!Unit.Type! corresponding to a given \verb!String!
\end{itemize}
and many more.


\section{Replacement of the interface \texttt{libSBMConstants} by Java enums}

You won't find a corresponding implementation of this interface in 
JSBML. The reason is that the JSBML team decided to encode constants using the
Java construct enum. For instance, all the fields starting with the
prefix \verb!AST_TYPE_*! have a corresponding field in the \verb!ASTNode! class
itself. There you can find the Type enum. Instead of typing
\verb!AST_TYPE_PLUS!, you would therefore type \verb!ASTNode.Type.PLUS!.

The same holds true for \verb!Unit.Kind.*! corresponding to the 
\verb!LibSBMLConstants.UNIT_KIND_*! fields.


\section{Various types of \texttt{ListOf*} classes}

There is no method \verb!get(String id)! because the generic implementation of 
the \verb!ListOf<? extends SBase>! class in JSBML excepts also elements that do 
not necessarily have an identifier. Only instances of \verb!NamedSBase! may have
the fields identifier and name set. Hence, generally, the \verb!ListOf! class 
cannot assume these fields to be present. To query an instance of \verb!ListOf! 
in JSBML for names or identifiers or both, you can apply the following 
filter:
\begin{verbatim}
NamedSBase nsb = myList.firstHit(new NameFilter(identifier));
\end{verbatim}
This will give you the first element in the list with the given identifier.
Various filters are already implemented, but you can easily add your 
customized filter. To this end, you only have to implement the \verb!Filter! 
interface in \verb!org.sbml.jsbml.util.filters!. There you can also find an
\verb!OrFilter! and an \verb!AndFilter!, which take as arguments multiple other
filters. With the \verb!SBOFilter! you can query for certain SBO annotations in
your list, whereas the \verb!CVTermFilter! helps you to identify \verb!SBase!
instances with a desired MIRIAM annotation. For instances of
\verb!ListOf<Species>! you can apply the \verb!BoundaryConditionFilter! to look
for those species that operate on the boundary of the reaction system.


\section{Units}

Since SBML Level 3 the data type of the exponent attribute in the \verb!Unit! class
has been changed from \verb!int! to \verb!double! values. JSBML reflects this in
the method \verb!getExponent()! by returning \verb!double! values only. For a
better compatibility with LibSBML, whose corresponding method still returns 
\verb!int! values, JSBML also provides the method \verb!getExponentAsDouble()!.
This method returns the value from the \verb!getExponent()! method and is
therefore absolutely redundant.

\section{Unit Definitions}

A model in JSBML always also contains all predefined units in the model
if there are any, i.e., for models encoded of SBML versions before level
3. These can be accessed from an instance of model by calling the method
\verb!getPredefinedUnit(String unit)!.


% \bibliographystyle{natbib}
% \bibliography{literature}

\end{document}

