\section{Deprecation}

The intension of JSBML is to provide a Java library for the latest 
specification of SBML. Hence, JSBML provides methods and classes to
cover earlier releases of SBML as well, but these are often marked
as being deprecated to avoid creating models that refer to these 
elements.

\section{UnitDefinitions}

A model in JSBML always also contains all predefined units in the model
if there are any, i.e., for models encoded of SBML versions before level
3. These can be accessed from an instance of model by calling the method
getPredefinedUnit(String unit).

\section{MathContainer}

This interface gathers all those elements that may contain mathematical
expressions encoded in abstract syntax trees (ASTNodes). The abstract
class AbstractMathContainer serves as actual super class for most of
the derived types.

\section{ASTNodeCompiler}

This interface allows users to create customized interpreters for the
content of mathematical equations encoded in abstract syntax trees. It
is directly and recursively called from the ASTNode class and returns
an ASTNodeValue object, which wraps the possible evaluation results of
the interpretation. JSBML already provides several implementations of
this interface, for instance, ASTNodes can be directly translated to
LaTeX or MathML for further processing. 

\section{InitialAssignment}

JSBML unifies all those elements that assign values to some other 
SBase in SBML under the interface Assignment. This interface uses
the term Variable for the element whose value is to be changed depending
on some mathematical expression that is also present in the Assignment
(because Assignment extends the interface MathContainer). Therefore,
an Assignment contains methods such as set/getVariable(Variable v)
and also isSetVariable() and unsetVariable(). In addition to that
JSBML also provides the method set/getSymbol(String symbol) in the
InitialAssignment class to make sure that switching from libSBML to
JSBML is quite smoothly. However, the preferred way in JSBML is to
apply the methods setVariable either with String or Variable instances
as arguments.
 

\section{The class libSBML}

There is no class libSBML because this library is called JSBML. You
can therefore only find a class JSBML. This class provides similar
methods as the libSBML class in libSBML.

\section{libSBMLConstants}

You won't find a corresponding implementation of this interface in 
JSBML. The reason is that we decided to encode constants using the
Java construct enum. For instance, all the fields starting with the
prefix AST_TYPE_* have a corresponding field in the ASTNode class itself.
There you can find the Type enum. Instead of typing AST_TYPE_PLUS, you
would therefore type ASTNode.Type.PLUS.

The same holds true for Unit.Kind.* corresponding to the 
libSBMLConstants.UNIT_KIND_* fields.

\section{ListOf}

There is no method get(String id) because the generic implementation of 
the ListOf<? extends SBase> class in JSBML excepts also elements that do 
not necessarily have an identifier. Only instances of NamedSBase may have
the fields identifier and name set. Hence, generally, the ListOf class 
cannot assume these fields to be present. To query an instance of ListOf 
in JSBML for names or identifiers or both, you can apply the following 
filter:

NamedSBase nsb = myList.firstHit(new NameFilter(identifier));

This will give you the first element in the list with the given identifier.
Various filters are already implemented, but you can easily add your 
customized filter. To this end, you only have to implement the Filter 
interface in org.sbml.jsbml.util.filters. There you can also find an
OrFilter and an AndFilter, which take as arguments multiple other filters.
With the SBOFilter you can query for certain SBO annotations in your list,
whereas the CVTermFilter helps you to identify SBase instances with a
desired MIRIAM annotation. For instances of ListOf<Species> you can apply
the BoundaryConditionFilter to look for those species that operate on the
boundary of the reaction system.



