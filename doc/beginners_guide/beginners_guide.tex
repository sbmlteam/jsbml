\documentclass{JSBMLdoc}

\newcommand{\jsbmlversion}{1.2}
\newcommand{\docdate}{\today}

\makeindex

%%%%%%%%%%%%%%%%%%%%%%%%%%%%%%%%%%%%%%%%%%%%%%%%%%%%%%%%%%%%%%%%%%%%%
% 
% A collection of useful macros that facilitat maintaining the JSBML
% documentation.
%
% Author: Andreas Dr\"ager
%%%%%%%%%%%%%%%%%%%%%%%%%%%%%%%%%%%%%%%%%%%%%%%%%%%%%%%%%%%%%%%%%%%%%

% A
\newcommand{\AbstractNamedSBase}{\texttt{AbstractNamedSBase}\index{SBase@\texttt{SBase}!\texttt{AbstractNamedSBase}}}
\newcommand{\ASTtype}{\texttt{AST\_TYPE\_*}\index{ASTNode@\texttt{ASTNode}!\texttt{AST\_TYPE\_*}}}
\newcommand{\ASTNode}{\texttt{ASTNode}\index{ASTNode@\texttt{ASTNode}}}

% C
\newcommand{\CallableSBase}{\texttt{CallableSBase}\index{JSBML!CallableSBase@\texttt{CallableSBase}}\index{SBase@\texttt{SBase}!CallableSBase@\texttt{CallableSBase}}}
\newcommand{\Compartment}{\texttt{Compartment}\index{SBML!Compartment@\texttt{Compartment}}}

% H
\newcommand{\History}{\texttt{History}\index{annotation!\texttt{History}}}

% K
\newcommand{\KineticLaw}{\texttt{KineticLaw}\index{KineticLaw@\texttt{KineticLaw}}}

% M
\newcommand{\Model}{\texttt{Model}\index{model!Model@\texttt{Model}}}
\newcommand{\ModelCreator}{\texttt{ModelCreator}\index{annotation!\texttt{ModelCreator}}}
\newcommand{\ModelHistory}{\texttt{ModelHistory}\index{annotation!\texttt{ModelHistory}}}

% N
\newcommand{\NamedSBase}{\texttt{NamedSBase}\index{SBase@\texttt{SBase}!\texttt{NamedSBase}}}

% S
\newcommand{\SBase}{\texttt{SBase}\index{SBase@\texttt{SBase}}}
\newcommand{\SBML}{SBML\index{SBML}}
\newcommand{\SBMLDocument}{\texttt{SBMLDocument}\index{SBML!SBMLDocument@\texttt{SBMLDocument}}}
\newcommand{\Serializable}{\texttt{Serializable}\index{Serializable@\texttt{Serializable}}}
\newcommand{\Species}{\texttt{Species}\index{SBML!Species@\texttt{Species}}}

% T
\newcommand{\TreeNode}{\texttt{TreeNode}\index{TreeNode@\texttt{TreeNode}}} 

% U
\newcommand{\UnitDefinition}{\texttt{UnitDefinition}\index{Unit!UnitDefinition@\texttt{UnitDefinition}}}

\hyphenation{
AST-Node
Boo-lean
Cell-De-sig-n-er
Call-able-S-Base
Event-Listener
Event-Hand-l-er
Event-Object
get-Ex-po-nent
get-Ex-po-nent-As-Double
get-Spatial-Di-men-sions
get-Spatial-Di-men-sions-As-Double
J-SB-ML
Ki-ne-tic-Law
lib-SBML-Constants
Lib-SB-ML-Reader
Local-Pa-ra-meter
Plug-in-Action
plug-in
Property-Change-Event
SBML-Document
Tree-Node
Tree-Node-Change-Event
Tree-Node-Change-Listen-er
Tree-Node-With-Change-Support
time-Units
Tree-Node
}


\hypersetup{
  bookmarksopenlevel={1},
  bookmarksnumbered={true},
  breaklinks={true},
  pdfpagemode={UseOutlines},
  pdftitle={User guide for JSBML},
  pdfauthor= {Thomas M. Hamm} {Nicolas Rodriguez} {Andreas Dr\"ager} {Michael Hucka},
  pdfsubject={Software guide},
  pdfkeywords={JSBML} {libSBML} {Java} {SBML} {API} {LaTeX} {documentation}
{manual} {guide} {code examples},
  pdfview={FitV},
  pdffitwindow={true},
  pdfstartview={FitV},
  pdfnewwindow={false},
  pdfdisplaydoctitle={true},
  pdfhighlight={/P},
  plainpages={false},
  unicode={true}
}

\definecolor{dkgreen}{rgb}{0,0.6,0}
\definecolor{gray}{rgb}{0.5,0.5,0.5}
\definecolor{mauve}{rgb}{0.58,0,0.82}

\lstset{ %
  language=Java,                  % the language of the code
%  basicstyle=\footnotesize,       % the size of the fonts that are used for the code
  numbers=left,                   % where to put the line-numbers
  numberstyle=\tiny\color{gray},  % the style that is used for the line-numbers
  stepnumber=1,                   % the step between two line-numbers. If it's 1, each line 
                                  % will be numbered
  numbersep=5pt,                  % how far the line-numbers are from the code
  backgroundcolor=\color{white},  % choose the background color. You must add \usepackage{color}
  showspaces=false,               % show spaces adding particular underscores
  showstringspaces=false,         % underline spaces within strings
  showtabs=false,                 % show tabs within strings adding particular underscores
  frame=single,                   % adds a frame around the code
  rulecolor=\color{black},        % if not set, the frame-color may be changed on line-breaks within not-black text (e.g. commens (green here))
  tabsize=4,                      % sets default tabsize to 2 spaces
  captionpos=b,                   % sets the caption-position to bottom
  breaklines=true,                % sets automatic line breaking
  breakatwhitespace=false,        % sets if automatic breaks should only happen at whitespace
  title=\lstname,                 % show the filename of files included with \lstinputlisting;
                                  % also try caption instead of title
  keywordstyle=\color{blue},          % keyword style
  commentstyle=\color{dkgreen},       % comment style
  stringstyle=\color{mauve},         % string literal style
  escapeinside={|}{|},            % if you want to add a comment within your code
  morekeywords={*,...}               % if you want to add more keywords to the set
}

% -----------------------------------------------------------------------------
\begin{document}

\title{\textls[20]{JSBML Beginners Guide}}

\version{\jsbmlversion\\[0.5em]{\normalsize\emph{Document build date: \docdate}}}

\newcommand{\where}[1]{\,\textsuperscript{#1}}
\newcommand{\divider}[1]{\multicolumn{2}{c}{\emph{#1}:}}

\author{
  \setlength{\tabcolsep}{20pt}
  \begin{tabular}{@{}cc@{}}
    \divider{Authors}\\[0.75em]
    Thomas M. Hamm\where{a} & Nicolas Rodriguez\where{b} & Andreas Dr\"ager\where{a} & Michael
    Hucka\where{c}\\[2em]
    \divider{Institutional affiliations}\\[0.5em]
  \end{tabular}
  \\
  \begin{normalsize}
  \where{a\,}Computational Systems Biology of Infection and Antimicrobial-Resistance, University of T\"ubingen, T\"ubingen, Germany
  \\[0.25em]
  \where{b\,}The Babraham Institute, Babraham Campus, Cambridge, UK
  \\[0.25em]
  \where{c\,}Computing and Mathematical Sciences, California Institute of Technology, Pasadena, CA,
  USA
  \end{normalsize}
}

\frontNotice{SBML (the Systems Biology Markup Language) is an XML-based
  model representation format for storing and exchanging computational
  descriptions of biological processes.  To read, write, manipulate, and
  perform higher-level operations on SBML files and data streams, software
  applications need to map SBML entities to suitable software objects.
  JSBML provides a pure Java library for this purpose.  It supports all
  Levels and Versions of SBML and provides many powerful features,
  including facilities to help migrate from the use of libSBML (a popular
  library for SBML that is not written in Java).
  \\ \\
  This document provides an introduction to JSBML and its use.  It is aimed
  at both developers writing new Java-based applications as well as those
  who want to adapt libSBML-based applications to using JSBML.  This user
  guide is a companion to the JSBML API documentation.
  \\ \\
  \centerline{The JSBML home page is \url{\jsbmlHomePageURL}.}\\
  \centerline{The JSBML discussion group is \url{\jsbmlGroupURL}.}  }


\maketitlepage
\maketableofcontents
\clearpage


\chapter{Getting started with JSBML}
\label{chp:getting-started}

Getting started with JSBML \cite{Draeger2011a, Draeger2011b}


\chapter{Acknowledgments}
\label{chp:acknowledgements}


% -----------------------------------------------------------------------------
% Appendix
% -----------------------------------------------------------------------------

\appendix

\chapter{Frequently Asked Questions (FAQ)}
\label{chp:faq}


\chapter{Open tasks in JSBML development}
\label{chp:open-tasks}


% -----------------------------------------------------------------------------
% References
% -----------------------------------------------------------------------------

\clearpage

\thispagestyle{plain}
\pagestyle{plain}
\bibliography{literature}

% -----------------------------------------------------------------------------
% Index
% -----------------------------------------------------------------------------

\setindexprenote{\vspace*{0.1ex}}
\printindex


\end{document}
