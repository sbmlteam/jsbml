% -*- TeX-master: "User_guide"; fill-column: 75 -*-

\section{JSBML modules}
\label{sec:jsbml-modules-details}

JSBML modules extend the functionality of JSBML and are provided as
separate libraries (packaged as JAR files). With the help of the current
JSBML modules, JSBML can be used, for example, as a communication layer
\index{JSBML!as communication layer} between your application and
libSBML~\citep{Bornstein2008} or between your program and the program known
as CellDesigner~\citep{Funahashi2003}. In addition, JSBML is planning to offers a
compatibility module \index{libSBML!compatibility module}%
that helps to write code compatible with libSBML by providing the same package 
structure and API as libSBML's Java
language interface. In the rest of this section, we provide examples of how
to use these modules.


\subsection{The \codeNC{libSBMLio} module: using libSBML for parsing
  SBML into JSBML data structures}

The capabilities of the SBML validator\index{SBML!validator} constitutes one
of the major strengths of libSBML~\citep{Bornstein2008} in comparison to
JSBML, which does not yet contain a standalone validator for SBML, but makes
use of the online validation provided at \url{http://sbml.org}. However, if
the platform-dependency of libSBML does not hamper your application, or you
want to switch slowly from libSBML to JSBML, you may still read and write
SBML models using libSBML in conjunction with JSBML.

To facilitate this, the module \code{libSBMLio} provides the classes
\LibSBMLReader and \LibSBMLWriter.  \fig{lst:LibSBMLio} provides a short
code example illustrating the use of \LibSBMLReader.  The program displays the
content of an SBML file in a \JTree, similar to what is shown in
\fig{fig:JSBMLvisualizer-output}.

As of version 1.0 of JSBML, the \code{libSBMLio} module also contains
specialized \TreeNodeChangeListener{}s that synchronize any change in the
JSBML data structure with corresponding libSBML data structures.

\begin{figure}[htb]
  \exampleFile[style=java, caption={}, firstline=33]{src/org/sbml/jsbml/xml/libsbml/libSBMLio_example.java}
  \caption{A simple example showing how to convert libSBML data structures
    into JSBML data objects.  To run this example, you need libSBML
    installed on your system.  You may need to set environment variables,
    e.g., the \code{LD\_LIBRARY\_PATH}
    \index{libSBML!LD\_LIBRARY\_PATH@\code{LD\_LIBRARY\_PATH}}%
    under Linux\index{operating system}, to values appropriate for your
    system. For details, please see the libSBML
    documentation~\cite{libSBMLwebsite}.}
  \label{lst:LibSBMLio}
\end{figure}


\subsection{\textls[-20]{The \codeNC{CellDesigner} module: turning a
    JSBML-based application into a CellDesigner plugin}}

Once an application has been implemented based on JSBML, it can easily be
accessed from CellDesigner's plugin menu~\citep{Funahashi2003}. To support
this, it is necessary to extend two classes that are defined in
CellDesigner's plugin API. 
\index{application programming interface!CellDesigner}
\vrefrange{lst:PluginAction}{lst:Plugin1}
show a simple example of (1) how to pass a model\index{model!CellDesigner}
data structure in a CellDesigner plugin to the translator in JSBML, and (2)
\index{CellDesigner!plugin} \index{CellDesigner!\code{PluginAction}}%
creating a plugin for CellDesigner which displays the SBML data structure
in a tree, like the example in \fig{fig:JSBMLvisualizer-output}.

\clearpage
\exampleFile[style=java, firstline=29, caption={A simple implementation of CellDesigner's abstract class \code{PluginAction}.}, label={lst:PluginAction}]{src/org/sbml/jsbml/celldesigner/cdplugin_example.java}

Listings \vrefrange{lst:PluginAction}{lst:Plugin1} show how to translate a
plugin's data structure from CellDesigner into a JSBML data structure. With
the help of the class \PluginSBMLWriter, it is possible to notify
CellDesigner about changes in the data structure. Note that the program in
listing \fig{lst:Plugin1} is only completed by implementing the methods from the
superclass, \code{CellDesignerPlugin}; it is sufficient to leave the
implementation empty.

As of JSBML version 1.0, this module also contains a specialized
implementation of the \TreeNodeChangeListener{} interface for
synchronization of changes in JSBML's data structures with CellDesigner.

\clearpage

\exampleFile[style=java, firstline=35, label=lst:Plugin1, caption={A simple example for a CellDesigner plugin using JSBML as a communication layer.}]{src/org/sbml/jsbml/celldesigner/SimpleCellDesignerPlugin.java}

\clearpage

\subsection{The \codeNC{libSBMLcompat} module: a JSBML compatibility
  module for libSBML}

The goal of the libSBML compatibility module in JSBML is to provide the
same package structure as libSBML's Java bindings, and provide
identically-named classes and APIs. Using the module, it will be possible
to switch an existing application from libSBML to JSBML or the other way
around without changing any code.  \index{libSBML!compatibility module}%
This module is in early development phase.

\subsection{The \codeNC{android} module: a compatibility module for
  Android systems}
\index{Android}

The JSBML \emph{Android} module is intended to provide all those classes
from the Java standard distribution that are required for JSBML, but might
be missing on Android systems.

\subsection{The \codeNC{compare} module: facilities for doing comparisons between libSBML and JSBML}

During the early development of JSBML, we developed those set of classes in order to check that what
JSBML was reading in memory was equivalent of what libSBML was reading in memory. Those classes were
used (and can still be used) to detect inconsistency between JSBML and libSBML and helped to find bugs
 in both libraries.

\subsection{The \codeNC{tidy} module: to produce a tidy XML output}

The tidy module was created to allow users to write a pretty XML output. In order to use it, you just need to replace
in your code the use of the \SBMLWriter{} class by the \TidySBMLWriter{} class.
