% -*- TeX-master: "User_guide"; fill-column: 75 -*-

\section{Change listeners}

JSBML offers the ability to listen to change events in the life of an SBML
document. To benefit from this facility, simply let your class implement
the interface \TreeNodeChangeListener and add it to the list of listeners
in your instance of \SBMLDocument. You only have to implement three
methods:

\begin{description}[font=\normalfont]

\item[\code{nodeAdded(TreeNode node)}:] This method notifies the listener
  that the given \TreeNode instance has just been added to the \SBMLDocument
  object. When this method is called, the given node is already fully linked
  to the \SBMLDocument, i.e., it has a valid parent that in turn points to
  the given node.

\item[\code{nodeRemoved(TreeNodeRemoveEvent evt)}:] This method notifies the listener
  that a \TreeNode has just been removed, and therefore is
  no longer part of the \SBMLDocument. The deleted element can be accessed 
  using the \code{getSource()} method of the given event object. The 
  \SBMLDocument will no longer contain
  pointers to this node; however, the event object will contain
  a pointer to its former parent, and it can be accessed by calling \code{getPreviousParent}
  on the event object. (This makes it possible to recognize where
  in the tree this node was located and even to revert the deletion of the
  node.)

\item[\code{propertyChange(PropertyChangeEvent evt)}:] This method
  provides detailed information about the change in a value within the
  \SBMLDocument.  The object passed to this method is a
  \TreeNodeChangeEvent, which provides information about which \TreeNode
  has been changed, which of its properties has been changed (as a
  \String representation of the name of the property), the previous value,
  and the new value.

\end{description}

These methods can help software track what their \SBMLDocument objects are
doing at any given time.  Furthermore, these features can be very useful in
a graphical user interface\index{graphical user interface}, where, for
example, the user might need to be asked if he or she really wants to
delete some element or to approve changes before making these
persistent. Another way this can be used is for writing log
files\index{logging!log file} of the model-building \index{model} process
automatically. To this end, JSBML already provides the implementation
\SimpleTreeNodeChangeListener which notifies a logger about each change.

Note that the class \TreeNodeChangeEvent extends the class
\code{java.beans.Property\-Change\-Event},
\index{event!PropertyChangeEvent@\code{PropertyChangeEvent}} which is
derived from
\code{java.util.EventObject}\index{event!EventObject@\code{EventObject}}.
It should also be pointed out that the interface \TreeNodeChangeListener
extends the interface \code{java.beans.Pro\-perty\-Change\-Listener}
\index{event!PropertyChangeListener@\code{PropertyChangeListener}} which in
turn extends the interface \EventListener in the package
\code{java.util}. In this way, the event and listener data structures fit
into common Java API idioms \index{application programming interface!Java}
and allow users also to make use of, e.g., \EventHandler{}s to deal with
changes in an SBML model\index{model}.

As mentioned in \sec{sec:AbstractTreeNode}, all major objects implement
the interface \TreeNode, and its listeners are notified about all changes
that occur in any implementing data structure. The use of
\TreeNodeChangeListener{}s allows a software application not only to keep
track of changes in instances of \SBase, but also changes inside of, e.g.,
\CVTerm or \History.
