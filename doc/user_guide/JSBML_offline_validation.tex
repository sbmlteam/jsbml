% -*- TeX-master: "User_guide"; fill-column: 75 -*-

\section[Offline validation]{The JSBML offline validator}

The JSBML \emph{offline validator} is a self-contained SBML validation facility that implements some of the validation processes included in libSBML.  At the time of this writing, \textbf{the offline validator is incomplete and unsuitable for production use}.  We included it here so that users can experiment with it.  We hope to improve this facility so that in time it can be used to validate SBML files reliably.


\subsection{Basic procedure for using offline validation in JSBML}

The following sections describe the basic steps required to use the offline
validator in your code.


\subsubsection{Create an instance of \codeNC{ValidationContext}}

\code{ValidationContext} is the center of the validation process.  The
constructor for this class requires two arguments, for the Level and Version
of SBML you want to validate. These values can be changed after creating the
context object, but note that doing so will clear any loaded constraints
(discussed below).

\vspace*{1ex}
\begin{example}[style=java, title={Setup a ValidationContext.}]
// Create a new instance
ValidationContext ctx = new ValidationContext(3, 1);
\end{example}

The \code{ValidationContext} is designed to be reusable, so you can use a
single instance of ValidationContext to perform several validations.  To do
so, repeat the next two steps.


\subsubsection{Prepare the context}

Next, provide \code{ValidationContext} with the JSBML SBML objects that you
want to validate. For each such object, use the method
\code{loadConstraintsForClass(Class<?> clazz)} with the objects class as
parameter (e.g., ``\code{object.getClass()}") or by using a static reference
(e.g., ``\code{Species.class}"). The context will traverse the class
hierarchy to load all the constraints that are necessary to achieve
validation. For example, if you use a custom class which is derived from
\Species, the validator will recognize this and also load the constraints for
the \Species class.

\vspace*{1ex}
\begin{example}[style=java, title={Three different ways to load constraints.}]
// Load constraints to validate a MyClass object
ctx.loadConstraintsForClass(MyClass.class);	

// Load constraints to validate the class from myObject
ctx.loadConstraintsForClass(myObject.getClass());

// Load constraints to validate a single attribute
ctx.loadConstraintsForAttribute(myObject.getClass(), "name");

// NOTICE: the loadConstraints methods clears the root constraint.
// After the third command, the context will only contain the constraints to check the "name" attribute.
\end{example}


\subsubsection{Run the validation}

Finally, start the validation procedure. The method \code{validate(Object o)}
will return a Boolean whose value will be \code{true} when no constraint was
broken and \code{false} otherwise.

\begin{example}[style=java, title={Validate.}]
// Perform the validation
boolean isValid = ctx.validate(myObject);
\end{example}

If you invoke the \code{validate} method on an object that is not assignable
to the class for which the constraints were loaded, this method will simply
return \code{false} and print a log message.  To get a list of all the errors
and broken constraints, use a instance of \code{LoggingValidationContext}
instead.


\subsubsection{Control the validation}

There are several ways to control the behavior of the validation process.

\begin{enumerate}

\item \emph{Enable/disable check categories}.  With check categories,
  it's possible to control which subset of rules will be loaded. 

\item \emph{Recursive validation}.  The validation context can perform a
  recursive validation and also validate the child objects. This is realized
  by using the \TreeNode Interface. If a class inherits from \TreeNode (which
  is the case by \SBase) and this option is enabled, the context will also
  load the constraints for the class of every child returned by the \TreeNode
  iteration methods. If one of the children is also a \TreeNode, the
  recursive validation will go on. This option is enabled by default.

\item \emph{Track the validation.}  To follow the validation process in real
  time, use the \code{ValidationListener} interface.  It provides two
  methods:

  \begin{itemize}

  \item \code{willValidate(ValidationContext ctx, AnyConstraint<?> c, Object o)} \\ 
    which is called every time before a constraint will perform a check

  \item \code{didValidate(ValidationContext ctx, AnyConstraint<?> c, Object o, boolean success)} \\
    which is called after the check. The \code{boolean success} will be the result of the check from this constraint.

  \end{itemize}

  If you want to get more information about the constraint, you can retrieve
  its error code. If the constraint is just a \code{ConstraintGroup} and
  therefore just a collection of constraints, the error code should be equals
  to \code{CoreSpecialErrorCodes.ID-GROUP}. In any other case you could use
  this error code to create the \code{SBMLError} object assoscieted with it
  by using the \code{SBMLErrorFactory}.

\end{enumerate}


\subsection{Providing custom constraints to the offline validator}

It is easy to provide custom constraints. When the \code{ConstraintsFactory}
looks for the constraints for a given class, it uses Java reflection to
search for a constraint declaration for that class. A constraint declaration
is just a simple class that has the name of the class it wants to declare
constraints for, followed by the word ``Constraints''.  For example, the
class that provides the constraints for \Species is called
\code{SpeciesConstraints}. This class must at least implement the
\code{ConstraintDeclaration} interface, but it is best if it also extends
\code{AbstractConstraintDeclaration} because the latter implements most
functions and also caches the constraints.  Notice that any constraint
declaration must be located in the package
``\code{org.sbml.jsbml.validator.offline.constraints}''.

To declare new constraints, follow these steps:

\begin{enumerate}

\item \emph{Create constraint declaration}. First you have to create the
  class in which you put the code for the constraints and select which
  constraints should be loaded to validate a certain check category.

\begin{example}[style=java, title={Create constraint declaration class}]
// Be sure to use this package, otherwise the ConstraintFactory won't find your constraints.
package org.sbml.jsbml.validator.offline.constraints;

// This class will contain the constraints for a MyClass object
public class MyClassConstraints extends AbstractConstraintDeclaration {
	
	// To be filled with your code...
}
\end{example}

\item \emph{Select the error codes which should be checked}.  Next, you
  have to collect the error codes to perform a proper validation. There are
  two methods, one for the complete validation in a single check category and
  one for the attribute validation. Inside this methods you will have a
  \code{Set<Integer>} to which you should add every error code that should be
  validated in this check category. You could use the level and version
  parameter to avoid loading unnecessary constraints.

\begin{example}[style=java, title={Select error codes}]
  /*
   * In this method you add all the error codes to the set, which should be 
   * covered for the given combination of check category, level and version.
   */
  public void addErrorCodesForCheck(Set<Integer> set, int level, int version,
    CHECK_CATEGORY category) {
    
    switch (category) {
    case GENERAL_CONSISTENCY:
    	// A helper function to add a range to the set (including the last one)
    	addRangeToSet(set, 6, 9);
    		
    	if (level > 1)
    	{
    		set.add(15);
    	}
    
    // other cases...
  	
    }
  }

  /*
   * This method works just like the one above, expect that this time you should
   * collect the error codes to validate only a single attribute of a object.
   *
   * Because the attribute validation is used to catch invalid values in the setters,
   * you should only select error codes which has severity "ERROR" in the given
   *  level and version.
   */
  public void addErrorCodesForAttribute(Set<Integer> set, int level,
    int version, String attributeName) {
	switch (category) {
    case "name":
    	set.add(8);
    	break;
    	
    // other cases...
    
    }
  }
\end{example}

\item \emph{Provide the logic for the constraints}.  Now you only have to
  write the constraint, this is done by provding a validation function. In
  the most cases you will just create a anonymus class and put your code in
  the \code{check(*)} method. This method should return \code{false} if the
  constraint is broken. Keep in mind that the constraints are cached and will
  be reused and shared between different \code{ValidationContext} objects.

  You could avoid caching by using negative numbers as your error codes.
  Beside of this, it's possible (and sometime necessary) to have different
  constraints with the same error code. Choose your error codes wisely,
  because classes like the \code{LoggingValidationContext} will use the error
  code to load a \code{SBMLError} object for this constraint.

\begin{example}[style=java, title={Select error codes}]
  /*
   * Here you provide the real logic for a constraint in form of a ValidationFunction
   * these function should return true if everything is fine and false otherwise.
   */
  public ValidationFunction<?> getValidationFunction(int errorCode) {
  	ValidationFunction<MyClass> func = null;
  	
  	switch (errorCode){
  	case 4:
  		func  = new ValidationFunction<MyClass> {
  		
  			public boolean check(ValidationContext ctx, MyClass mc){
  			
  				//If there is a name...
  				if (mc.isSetName())
  				{
  					// it shouldn't be empty.
  					return !mc.getName().isEmpty();
  				}
  				return true
  			}
  		};
  		break;
  		
  	// the other cases...
  	
  	}
  	
    return func;
  }
\end{example}

\end{enumerate}