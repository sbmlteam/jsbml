% -*- TeX-master: "User_guide"; fill-column: 75 -*-

\section{Other utility classes provided by JSBML}

JSBML also provides additional utility classes besides those mentioned
above. In the paragraphs below, we describe some of these classes in more
detail.  All of them are gathered in the package
\code{org.sbml.jsbml.util}, where you can also find a growing number of
additional helpful classes.

\subsection{Mathematical functions and constants}

The class \code{org.sbml.jsbml.util.Maths}
\index{JSBML!Maths@\code{Maths}}%
contains several static methods for mathematical operations not provided by
the standard Java class \code{java.lang.Math}. Most of these methods are
basic operations, for instance, \code{cot(double x)} or \code{ln(double
  x)}.  The JSBML class \code{Maths} also provides some less commonly used
methods, such as \code{csc(double x)} or \code{sech(double x)} as well as
\code{double} constants representing Avogadro's number and the universal
gas constant
$R = 8.314472\;\mathrm{J}\cdot\mathrm{mol}^{-1}\cdot\mathrm{K}^{-1}$.
In this way, the functions and constants implemented in class \code{Maths}
complement standard Java with methods and numbers required by the SBML
specifications~\citep{Hucka2003, Hucka2008, Hucka2010a}.

\subsection{Some tools for \codeNC{String} manipulation}

The JSBML class \StringTools provides several methods for convenient
\String manipulation. These methods are particularly useful when parsing or
displaying \code{double} numbers in a \code{Locale}\hyp{}dependent way. To
this end, this class predefines a selection of useful number formats. It
can also wrap \String elements into HTML code, mask non-ASCII characters
using corresponding HTML codes, efficiently concatenate \String{}s, or
deliver the operating system-dependent new line character.

