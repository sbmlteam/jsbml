% -*- TeX-master: "User_guide"; fill-column: 75 -*-

\section{Differences between the APIs of JSBML and libSBML}
\label{sec:api-differences}

We strove to make JSBML be closely compatible with libSBML. However,
because of the different programming languages used,
some differences are impossible to overcome.
In other cases, an exact translation from libSBML's C/C++ \index{C}
code to Java would be inelegant and unnatural for Java users,
conflicting with another important goal of JSBML: to provide
an API \index{application programming interface!Java} whose classes and
methods behave, and are organized like, those in other Java libraries.

In this section, we discuss the most important differences in the APIs of
JSBML \index{application programming interface!JSBML} and libSBML.
\index{application programming interface!libSBML} We also provide some
examples of how the classes and methods in JSBML may be used.


\subsection{Level and Version \codeNC{ValuePair}}

In libSBML, the Level and Version information is recorded as individual
integers; by contrast, in JSBML it is stored in a generic object, \ValuePair,
stored within an \AbstractSBase instance. The class \ValuePair implements the
Java interface \Comparable and takes two values of any type that both also
implement \Comparable.  Storing the information in this way allows users to
check for a specific Level/Version combination more naturally, as the example
in \fig{fig:LevelVersionCheck} demonstrates. The method
\code{getLevelAndVersion()} in \AbstractSBase delivers an instance of
\ValuePair with the Level and Version combination for the respective element.

\begin{figure}[bh]%
  \begin{example}
if (mySBase.getLevelAndVersion().compareTo(Integer.valueOf(2), Integer.valueOf(2)) < 0) {
  throw new IllegalArgumentException("Cannot create a " + mySBase.getElementName() + 
  		" with Level = " + getLevel() + " and Version = " + getVersion() + ".");
}\end{example}
  \caption{Example program fragment showing how to check for a minimal
    expected SBML Level/Version combination.}
  \label{fig:LevelVersionCheck}
\end{figure}


\subsection{Abstract syntax trees for mathematical formulas}

Both libSBML and JSBML define a class called \ASTNode for in-memory storage
and evaluation of abstract syntax trees (ASTs) that represent mathematical
formulas. These can be parsed either from \String{}s containing formulas in a
C-like infix syntax, or from a MathML \index{MathML} representation.  JSBML's
\ASTNode class provides various methods to transform ASTs to other formats,
for instance, \code{String}s in \LaTeX \index{LaTeX@\LaTeX} syntax.  Several static methods also
make it easy to create syntax trees.  The next example creates a new
\ASTNode which represents the sum of the two other nodes:

\begin{example}
ASTNode myNode = ASTNode.sum(myLeftAstNode, myRightASTNode);
\end{example}

SBML specifies that mathematical formulas may contain references to the
following kinds of components in a model: \Parameter{}s,
\LocalParameter{}s, \FunctionDefinition{}s, \Reaction{}s, \Compartment{}s,
\Species, and in \SBMLthree, \SpeciesReference{}s.  In JSBML, all of these
classes implement a common interface, \CallableSBase, which extends
the interface \NamedSBaseWithDerivedUnit. This organization ensures that
only identifiers of these particular SBML components can be set in
instances of \ASTNode.


\subsubsection{Constructors and other methods for \CallableSBase}

JSBML provides useful constructors and methods to work with instances of
\CallableSBase.  The \code{set} method changes the type of an \ASTNode to
\ASTTypeName and directly sets the name to the identifier of the given
\CallableSBase.  The \code{get} method looks for the corresponding object in
the \Model and returns it. If no such object can be found or the type of the
\ASTNode is something different from \ASTTypeName, it throws an exception.

\begin{example}[style=java, title={Getter and setter for \CallableSBase.}]
public void setVariable(CallableSBase variable) { ... }

public CallableSBase getVariable() { ... }
\end{example}

The following are examples of methods for creating and manipulating complex
ASTs.  JSBML provides several static methods (such as \code{sum} shown above)
that create small trees from objects in memory.  Other methods, such as
\code{plus}, \code{frac} and \code{pow} change existing tree structures:

\begin{example}[style=java, title={Some examples of convenience methods, some of
    them static methods, provided by JSBML for working with \ASTNode{}s.}]
public ASTNode plus(CallableSBase nsb) { ... }

public static ASTNode frac(MathContainer container,
      CallableSBase numerator, CallableSBase denominator) { ... }

public static ASTNode pow(MathContainer container,
      CallableSBase basis, CallableSBase exponent) { ... }
\end{example}

In contrast to the static \code{ASTNode.sum} function at the beginning of
this section, the \code{frac} and the \code{pow} methods above take instances
of \CallableSBase as their arguments instead of \ASTNode objects. Consequently, the
parent \MathContainer must be passed to the methods in order to ensure that
valid data structures are created. (In case of methods that take \ASTNode
objects as arguments, such as the static \code{ASTNode.sum}, the parent
\MathContainer can be taken from the first given node object.)

Finally, with the following \ASTNode constructors, dedicated single nodes can
be created whose type (from the enumeration \ASTType) will be \code{NAME} and
whose name will be set to the identifier of the given \CallableSBase.

\begin{example}
public ASTNode(CallableSBase nsb) { ... }

public ASTNode(CallableSBase nsb, MathContainer parent) { ... }
\end{example}


\subsubsection{The \codeNC{ASTNodeCompiler} class}

JSBML provides the interface \ASTNodeCompiler; it allows users to create
customized interpreters for the contents of mathematical formulas encoded
in abstract syntax trees. It is directly and recursively called from the
\ASTNode class and returns an \ASTNodeValue object, which wraps the
possible evaluation results of the interpretation.  As alluded to above,
JSBML provides several implementations of this interface; for instance,
\ASTNode objects can be directly translated to C language-like \String{}s,
\LaTeX, \index{LaTeX@\LaTeX} or MathML \index{MathML} for further
processing.  In addition, the class \UnitsCompiler, which JSBML uses to
derive the unit of an abstract syntax tree, also implements this interface.


\subsection{Compartments}

In \SBMLthree~\citep{Hucka2010a}, the domain of the attribute
\code{spatialDimensions} on \Compartment is no longer $\lbrace 0, 1, 2,
3\rbrace$, which can be represented with a \code{short} value in Java, and is
instead a real-numbered value (i.e., a value in $\mathbb{R}$), which requires
a \code{double} value in Java. For this reason, the method
\code{getSpatialDimensions()} in JSBML always returns a \code{double}
value. For consistency with libSBML, the \Compartment class in JSBML also
provides the redundant method \code{getSpatialDimensionsAsDouble()} that
returns the identical value; it is marked as a deprecated method.
\index{compartment!\code{getSpatialDimensions()}}%
\index{compartment!\code{getSpatialDimensionsAsDouble()}}%


\subsection{Model history}

Before \SBML Level~3, only the \Model object could have an associated
history, that is, a description of the person(s) who build the model,
including names, email addresses, modification and creation dates.  In
Level~3 of SBML, it is possible to annotate every construct with a
history. This is reflected in JSBML by the name of the corresponding
object---\History---whereas it is named \ModelHistory in libSBML.  All
instances of \SBase in JSBML contain methods to access and manipulate its
\History. Also, JSBML does not have libSBML's classes \ModelCreator
and \ModelCreatorList because JSBML gathers its \Creator objects in a generic
\code{List<Creator>} in the \History.


\subsection{Units and unit definitions}
\label{sec:units}

There are differences between libSBML and JSBML's interfaces for handling
units.  We describe them next.


\subsubsection{The exponent attribute of units}

In \SBMLthree~\citep{Hucka2010a}, the data type of the exponent attribute
of a \Unit object changed from \code{int} in previous Levels to
\code{double} values. To provide a uniform interface no matter which Level
of SBML is being dealt with, JSBML's method \code{getExponent()} only
returns \code{double} values. In libSBML, \code{getExponent()} always
returns \code{int}, and there is an additional method,
\code{getExponentAsDouble()}, to handle the cases with \code{double}
values.  JSBML provides \code{getExponentAsDouble()} for compatibility with
libSBML, but it is a redundant method in JSBML's case and therefore is marked
as deprecated.
\index{unit!getExponent()@\code{getExponent()}}%
\index{unit!getExponentAsDouble()@\code{getExponentAsDouble()}}%


\subsubsection{Predefined unit definitions}

A model in JSBML \index{unit!predefined units} always contains all
predefined units defined by SBML.  These can be accessed from an instance
of \code{Model} by calling the method \code{getPredefinedUnit(String
  unit)}.

MIRIAM annotations\index{annotations!MIRIAM} \citep{Novere2005} have been
an integral part of SBML models since Level~2 Version~2\index{SBML!Level~2
  Version~2}. Recently, the \index{Ontology} Unit Ontology
(UO)~\citep{unitontology} \index{annotations!unit ontology} has been
included in the set of supported ontology and online resources of MIRIAM
annotations~\citep{Novere2005}. Since all the predefined units in SBML have
corresponding entries in the UO, JSBML \index{unit!MIRIAM annotation}%
automatically equips those predefined units with the correct MIRIAM URI in
form of a controlled vocabulary term (\code{CVTerm}) if the SBML
Level/Version combination of the model supports MIRIAM annotations.  In
addition, the \code{enum} \code{Unit.Kind}
\index{unit!Unit.Kind@\code{Unit.Kind}}%
also provides methods to directly obtain the entry from the UO that
corresponds to a certain unit kind and also contains methods to generate
MIRIAM URIs accordingly. In this way, JSBML facilitates the annotation of
user-defined units and unit definitions with MIRIAM-compliant
\index{annotations!MIRIAM} information.


\subsubsection{Access to the units of an element}

In JSBML, all SBML components whose value can be associated with a unit of
measurement implement the interface \SBaseWithUnit.  This interface provides
methods to access an object representing the unit. Currently, the interface
is implemented by \AbstractNamedSBaseWithUnit, \ExplicitRule, and
\KineticLaw.  \fig{fig:TypeHierarchy} provides an overview of the
relationships between these and other classes and interfaces.

% FIXME

\AbstractNamedSBaseWithUnit is the abstract superclass for \Event and
\QuantityWithUnit.  In the class \Event, all methods to deal with units are
deprecated because the \code{timeUnits} attribute was removed in SBML
Level~2 Version~2\index{SBML!Level~2}. The same holds true for instances of
\ExplicitRule and \KineticLaw which both can only be explicitly populated
with units in SBML Level~1\index{SBML!Level~1} for \ExplicitRule and before
SBML in Level~2, Version~3\index{SBML!Level~2} for \KineticLaw. By
contrast, the abstract class \QuantityWithUnit serves as the superclass for
\LocalParameter and \Symbol, which is then the superclass of \Compartment,
\Species, and (global) \Parameter.  With \SBaseWithUnit being a subclass of
\SBaseWithDerivedUnit, users can access the units of such an element in two
different ways:

\begin{description}[font=\normalfont]

\item[\code{getUnit()}:] This method returns a \String representation of
  the unit kind or the identifier of a unit definition in the model \index{model} that has
  been directly set by the user during the lifetime of the element. If
  nothing has been declared, this method returns an empty \String.

\item[\code{getDerivedUnit()}:] This method gives either the same
  result as \index{unit!derived unit} \code{getUnit()} if some unit has
  been declared explicitly, or it returns the predefined unit of the
  element for the given SBML Level/Version combination.  If neither a
  user-defined nor a predefined unit is available, this method returns an
  \index{String@\code{String}!empty} empty \String.

\end{description}

% FIXME

For convenience, JSBML also provides corresponding methods to the ones
above for directly obtaining an instance of \UnitDefinition.  However, care
must be taken when obtaining an instance of \UnitDefinition from one of the
classes implementing \SBaseWithUnit because it might happen that the
model\index{model} containing this \SBaseWithUnit does actually not contain
the required instance of \UnitDefinition and the method returns a
\code{UnitDefinition} that has just been created for convenience from the
information provided by the class. It may be useful for callers
to either check if the \Model{} contains this \UnitDefinition or to add it
to the \Model.

In the case of \KineticLaw, it is even more difficult because SBML
Level~1\index{SBML!Level~1} provides the ability to set the substance unit
and the time unit separately. To unify the API
\index{application programming interface!JSBML}, we decided to also provide methods that
allow the user to simply pass one \UnitDefinition or its identifier to
\KineticLaw.  These methods then try to guess if a substance unit or time
unit is given. Furthermore, it is possible to pass a \UnitDefinition
representing a variant of substance per time directly. In this case, the
\KineticLaw will memorize a direct link to this \UnitDefinition in the
model\index{model} and also try to save separate links to the time unit and
the substance unit. However, this may cause a problem if the containing
\Model does not contain separate \UnitDefinition{}s for both entries.


\subsection{Cloning when adding child nodes to instances of \codeNC{SBase}}

When adding elements such as a \Species to a \Model, libSBML \index{cloning}
will clone the object and add the clone to the \Model. In contrast, JSBML
does not automatically perform cloning. This has the advantage that
modifications on the object belonging to the original pointer will also
propagate to the element added to the \Model; furthermore, this is more
efficient at run-time and also more intuitive for Java programmers. If
cloning is necessary, users should call the \code{clone()} method
explicitly. Since all instances of \SBase, and also \Annotation, \ASTNode,
\CVTerm, and \History, extend \AbstractTreeNode (which in turn implements the
interface \Cloneable---see \fig{fig:TypeHierarchy}), all these elements can
be cloned naturally.  However, when cloning an object in JSBML
\index{cloning}, such as an \AbstractNamedSBase, all children of this element
will recursively be cloned before adding them to the new element. This is
necessary because the data structures specified in SBML
\index{SBML!hierarchical structure} define a tree, in which each element has
exactly one parent. It is important to note that some properties of the
elements must not be copied when cloning:

\begin{enumerate}

\item The pointer to the parent node of the top-level element that is
  recursively cloned is not copied and is left as \code{null} because the
  cloned object will get a parent set as soon as it is added or linked again
  to an existing tree. Note that only the top-level element of the cloned
  subtree will have a \code{null} value as its parent. All subelements will
  point to their correct parent element.

\item The list of \TreeNodeChangeListener objects is used in all other
  \code{setXX()} methods. Copying pointers to these may lead to
  unexpected behaviors, because during deep cloning, the listeners of
  the old object will suddenly be informed about all changes to values within
  the new object.  In cloning, all values of all child elements
  will be touched, i.e., all listeners will have to be informed many times, but
  each time they will receive the same value. Since they do not
  extend the \Cloneable interface, we cannot clone them either; for this reason, the cloned
  object has no \TreeNodeChangeListener object attached to it. The user is responsible
  for adding \TreeNodeChangeListener{}s on the cloned object if they want to be
  notified of any changes to it.

\item Since release 1.0, JSBML supports storing user objects in any object
  derived from \AbstractTreeNode.  These user objects are organized in a map
  data structure with object as key type, pointing to arbitrary user-defined
  objects. Note that, generally, deep cloning of these user objects is not
  possible, but JSBML keeps a pointer to these user objects in the cloned
  element.

\end{enumerate}


\subsection{Exceptions}
\label{sec:exceptions}

When errors occur, JSBML \index{exception} methods will usually throw an
exception whereas libSBML \index{exception!error codes} methods return a
numeric error code instead. The libSBML approach is rooted in the need to
support C-like languages, while exception handling is more natural in Java.
The JSBML approach of using exceptions helps programmers and users to avoid
creating invalid SBML data structures already when dealing with these in
memory. 

As per usual Java practice, JSBML methods declare that these may
potentially throw exceptions. In this way, programmers can be aware of
potential sources of problems already at the time of writing the source
code. Examples of the kinds of exceptions that JSBML methods may throw
include \ParseException, \index{exception!\code{ParseException}} which may
be thrown if a given formula cannot be parsed properly into an \ASTNode
data structure, and \InvalidArgumentException,
\index{exception!\code{InvalidArgumentException}} which may be thrown if
inappropriate values are passed to methods.

The following are some examples of situations that lead to exceptions:

\begin{itemize}

\item An object representing a constant\index{constant} such as a
  \Parameter whose \code{constant} attribute has been set to \code{true}
  cannot be used as the \Variable element in an \Assignment.
  \index{JSBML!assignment@\code{Assignment}}%
  \index{JSBML!variable@\code{Variable}}%
  \index{parameter!\code{Parameter}}%
  \index{parameter!\code{constant}}%

\item An instance of \Priority can only be assigned to an \Event{}s if its
  \code{level}\index{SBML!Level~3} attribute has at least been set to
  three.

\item Another example is the \InvalidArgumentException that is thrown when
  trying to set an invalid identifier \String for an instance of
  \AbstractNamedSBase.

\item JSBML keeps track of all identifiers within a model. For each
  namespace, it contains a separate map of identifiers within the \Model. It
  is therefore not possible to assign duplicate identifiers for
  elements that implement the interface \UniqueNamedSBase.  For
  \UnitDefinition{}s and \LocalParameter{}s separate maps are
  maintained. Since local parameters are only visible within the
  \KineticLaw that contain these, JSBML will only prohibit having more
  than one local parameter within the same list that has the identical
  identifier. All these maps are updated upon any changes within the
  model. When adding an element with an already existing identifier for its
  namespace, or changing some identifier to a value that is already defined
  within this namespace, JSBML will throw an exception.

\item ``Meta'' identifiers must be unique through the entire SBML file. To
  ensure that no duplicate meta-identifiers are created, JSBML keeps a map
  of all meta-identifiers on the level of the \SBMLDocument, which is
  updated upon any change of elements within the data structure. In this
  way, it is not possible to map the meta-identifier of some element to an
  already existing value or to add nodes to the SBML tree that contain a
  meta-identifier defined somewhere else within the tree. In both cases,
  JSBML will throw an exception. Since meta-identifiers can be generated in
  a fully automatic way (method \code{nextMetaId()} on
  \code{SBMLDocument}), users of JSBML should not care about these
  identifiers at all. JSBML will automatically create meta-identifiers
  where missing upon writing an SBML file.  (See \sec{sec:find-methods}.)

\item In case that spatial dimension units of a \Species{} are defined
  whose surrounding \Compartment{} has zero dimensions or that has only
  substance units, JSBML also throws an exception.

\end{itemize}

Hence, you have to be aware of potential exceptions and errors when using
JSBML, \index{exception} on the other hand, this will prevent you from doing
obvious mistakes. The class \SBMLReader in JSBML catches those errors and
exceptions. With the help of the logging utility, JSBML notifies users
about syntactical problems in SBML files. JSBML follows the rule that
illegal or invalid properties are not set.


\subsection{No interface  \codeNC{libSBMLConstants}}

JSBML does not contain an equivalent to libSBML's
\code{libSBMLConstants}. The reason is that in JSBML, constants are encoded
in a more natural Java fashion, \index{constant!\code{enum}}%
using the Java construct \code{enum}. For instance, all the fields starting
with the prefix \ASTTypePrefix{} have a corresponding field in the \ASTNode
class itself.   There you can find the enumeration \ASTType.  Thus, instead of typing
\code{libSBMLConstants.AST\_TYPE\_PLUS}, you type
\code{ASTNode.Type.PLUS}.  The same holds true for \code{Unit.Kind.*}
corresponding to the \code{libSBMLConstants.UNIT\_KIND\_*}
\index{unit!UNIT\_KIND\_*@\code{UNIT\_KIND\_*}}%
fields.

\subsection{No class \codeNC{libSBML}}

JSBML contains no class called \code{libSBML} simply because the library
is called \emph{JSBML}.  \index{libSBML!libSBML@\code{libSBML}}%
In its place, there is a class named \code{JSBML}.
\index{JSBML!JSBML@\code{JSBML}} This class provides some methods similar
to the ones provided in libSBML's \code{libSBML}, such as
\code{getJSBMLDottedVersion()} \index{JSBML!version}%
to obtain the current version of the JSBML library, which is 0.8 or 1.0-a* at the
time of writing this document. However, many other methods that you might
expect to find there, if you are used to libSBML, are located in the actual
classes that are related to the function. 

Here is an example of a method that is located in the relevant class.  To
convert between a \String \index{unit!String@\code{String}}%
and a corresponding \code{Unit.Kind}
\index{unit!Unit.Kind@\code{Unit.Kind}}%
you would use the following:

\begin{example}[style=java,
  title={Converting a string to a unit kind in JSBML.}]
Unit.Kind myKind = Unit.Kind.valueOf(myString);
\end{example}

Analogous to the above, the \ASTNode class provides a method to parse
C-like infix formula {\String}s according to the specification of SBML
Level~1~\citep{Hucka2003}\index{SBML!Level~1} into an abstract syntax
tree. Therefore, in contrast to the \code{libSBML} class, the class
\code{JSBML}\index{JSBML!JSBML@\code{JSBML}} contains only a few methods.


\subsection{No individual \codeNC{ListOf*} classes, but a generic \codeNC{ListOf}}

% We have the method get(String) on the ListOf and libsbml does not have it on
% the main ListOf class, only on subclasses where it is possible to do it.

JSBML does not have a specific \code{ListOf*}\index{ListOf*@\code{ListOf*}}
class for each type of \SBase elements, which is unlike the case in
libSBML. In JSBML, we use a generic implementation \code{ListOf<?~extends
  SBase>} that enables the same class to be used for each of the different
\code{ListOf*} classes defined in SBML while keeping a type-safe class.

To help developers work with \code{ListOf*} lists more conveniently, JSBML
provides several methods that use the Java \code{Filter} interface to search
and filter the lists. For example, to query an instance of a \code{ListOf*}
list in JSBML for specific identifiers, or names, or both, you can apply the
following filter:

\begin{example}[style=java,
  title={Example of searching a list for an object with a
    particular identifier.}]
NamedSBase nsb = myList.firstHit(new NameFilter(identifier));
\end{example}

This will return the first element in the list with the given identifier.  In
SBML, a \code{ListOf*} list object usually must not contain multiple elements
with the same identifier, so the element will usually be unique.  The
\code{firstHit} method stops after finding one element that satisfies the
given \code{Filter}. The \code{ListOf<?~extends SBase>} class also offers a
\code{filter} method that takes a \code{Filter} object as argument and
collects all elements accepted by that \code{Filter} object.

Various filters are already implemented in JSBML and made available for use
in your programs, but you can easily add your own custom filter. You only
need to implement the \Filter interface defined in the JSBML package
\code{org.sbml.jsbml.util.filters}.  In that package, you can also find an
\OrFilter and an \AndFilter, which take as arguments multiple other
filters. With the \SBOFilter you can query for certain SBO annotations
\citep{Novere2006, Novere2006b} \index{annotations!SBO}%
in your list; similarly, the \CVTermFilter helps you to identify \SBase
instances with a desired MIRIAM (Minimal Information Required In the
Annotation of Models) annotation~\citep{Novere2005}\index{annotations}. For
instances of \code{ListOf<Species>}, you can apply the
\BoundaryConditionFilter to look for those species\index{species!boundary
  condition} that operate on the boundary of the reaction system.


\subsection{Use of deprecation}

The intention of JSBML\index{JSBML!deprecation} is to provide a Java
library that supports the latest specifications of SBML\index{SBML}.
\index{SBML!specification}%
\index{deprecation}%
But we also want to support earlier specifications. So JSBML provides
methods and classes to cover elements and properties from earlier SBML
specifications as well, but these are often marked as being deprecated to
help users avoid creating models that refer to these elements. 

JSBML also contains many methods added for greater compatibility with
libSBML, but which programmers would probably not use unless they were
transitioning existing software from libSBML.  For instance, a method such
as \code{getNumXyz()} is not considered to be very Java-like (but such
methods are common for a C++\index{C++} programming style). Usually, Java
programmers would expect the method being called \code{getXyzCount()}
instead. For cases like this, JSBML provides alternative methods and marks
these methods that originate from libSBML as deprecated.
