% -*- TeX-master: "User_guide"; fill-column: 75 -*-

\section{Why are there differences?}
\label{sec:jsbml-design-goals}

In developing a pure Java Application Programming Interface (API) for
working with SBML, our intention was not to simply reimplement the Java API
already provided by libSBML~\citep{Bornstein2008}.
\index{application programming interface!libSBML}
We took the opportunity to rethink the API
\index{application programming interface!JSBML} from the ground up to
produce something more natural for Java programmers; moreover, we
benefited from being able to take a fresh look at today's entire set of
SBML specifications~\citep{Hucka2003, Hucka2008, Hucka2010a}
\index{SBML!specification} and redesign, for example, JSBML's type
hierarchy without the constraints of backward compatibility that libSBML
faces. 

JSBML has also been developed as a library that provides more than only
facilities for reading, manipulating, and writing SBML files and data
streams. Although SBML only defines the structure of representations of
biological processes in files and does not prescribe how its components
should be stored \emph{in computer memory}, many software developers
nevertheless find it convenient to follow similar representational
structures in their programs. \index{model!storage and exchange} With this
in mind, we designed JSBML with the intention that it be directly usable as
a flexible internal data structure for numerical computation,
visualization, and more. With the help of its \emph{modules}, JSBML can
also be used as a communication layer between applications. For instance,
JSBML facilitates the implementation of plugins for
CellDesigner~\citep{Funahashi2003}, \index{CellDesigner} a popular software
application for modeling and simulation in systems biology. Finally, JSBML
(like libSBML before it) hides some of the differences and inconsistencies
in SBML \index{SBML!differences between Levels} that grew into the language
over the years as it evolved from Level to Level and Version to Version;
this makes it considerably easier for developers to support multiple
Levels/Versions of SBML transparently.

Where possible, we maintained many of libSBML's naming conventions for
methods and variables. Owing to the very different backgrounds of the two
libraries, and the fact that libSBML is implemented in C \index{C} and
C++ \index{C++}, some differences are unavoidable. To help libSBML
developers transition more easily to using JSBML, we provide a
compatibility module that implements many libSBML methods as adaptors
around the corresponding JSBML methods.

% FIXME add back somehow:
% during
% the evolution of SBML some elements or properties of elements have become
% obsolete.
% \index{deprecation}%
