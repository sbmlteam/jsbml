\documentclass{bioinfo}
\copyrightyear{2011}
\pubyear{2011}

\usepackage{subfig}
\usepackage{listings}

% some nice colors
\definecolor{royalblue}{cmyk}{.93, .79, 0, 0}
\definecolor{lightblue}{cmyk}{.10, .017, 0, 0}
\definecolor{forrestgreen}{cmyk}{.76, 0, .76, .45}
\definecolor{darkred}{rgb}{.7,0,0}
\definecolor{winered}{cmyk}{0,1,0.331,0.502}
\definecolor{lightgray}{gray}{0.97}

\lstset{language=Java,
morendkeywords={String, SBMLsqueezer, LibSBMLReader, LibSBMLWriter, CfgKeys,
  Throwable}
captionpos=b,
basicstyle=\scriptsize\ttfamily\bfseries,
stringstyle=\color{darkred}\scriptsize\ttfamily,
keywordstyle=\color{royalblue}\bfseries\ttfamily,
ndkeywordstyle=\color{forrestgreen},
numbers=left,
numberstyle=\scriptsize,
% backgroundcolor=\color{lightgray},
breaklines=true,
tabsize=2,
frame=single,
breakatwhitespace=true,
identifierstyle=\color{black},
% morecomment=[l][\color{forrestgreen}]{//},
% morecomment=[s][\color{lightblue}]{/**}{*/},
% morecomment=[s][\color{forrestgreen}]{/*}{*/},
commentstyle=\color{forrestgreen}
% framexleftmargin=5mm,
% rulesepcolor=\color{lightgray}
% frameround=ttff
}

\begin{document}
\firstpage{1}
\application
\title[JSBML: The Java library for SBML]{JSBML: a flexible Java library
for working with SBML} \author[Dr\"ager \textit{et~al.}]{Andreas
Dr\"ager\,$^{1,*}$, Nicolas Rodriguez\,$^{2,*}$, 
Alexander D\"orr\,$^{1}$, 
Marine Dumousseau\,$^{2}$, 
Clemens Wrzodek\,$^{1}$, 
Nicolas Le Nov\`{e}re\,$^{2}$, 
Andreas Zell\,$^{1}$, 
Michael Hucka\,$^{3,*}$
\footnote{to whom correspondence should be addressed}}
\address{$^{1}$Center for Bioinformatics Tuebingen, University of Tuebingen, T\"ubingen, Germany.\\
$^{2}$European Bioinformatics Institute, Wellcome Trust Genome Campus, Hinxton, Cambridge, UK\\
$^{3}$Computing and Mathematical Sciences, California Institute of Technology, Pasadena, CA, USA}

\history{Received on XXXXX; revised on XXXXX; accepted on XXXXX}

\editor{Associate Editor: XXXXXXX}

\maketitle

\begin{abstract}

\section{Summary:}
The specifications of the Systems Biology Markup Language (SBML) define a standard for storing
and exchanging biochemical models in XML-formatted text files. To perform higher-level operations
on these models, e.g., numerical simulation or visual representation, an appropriate mapping to 
in-memory objects is required. To this end, the JSBML library has
been developed. In contrast to earlier approaches, JSBML has been especially designed for the
Java\texttrademark{} programming language and can therefore be used on all platforms, for which
a Java Runtime Environment is available. In this way, programs based on JSBML can, for instance, be 
easily released as Java webstart applications. JSBML's internal data structures
have been completely developed from scratch based on the definitions in the
SBML specifications but with respect to achieve the highest possible degree of
compatibility to the existing library libSBML. JSBML supports all SBML levels
and versions that are available today. In addition, JSBML provides modules that facilitate
the development of CellDesigner plugins or ease the migration from a libSBML backend.
\section{Availability:} Source code, binaries, and documentation of JSBML can 
be downloaded under the terms of LGPL 2.1 at \href{http://sbml.org/Software/JSBML}{http://sbml.org/Software/JSBML}.
\section{Contact:} \href{mailto:jsbml-team@sbml.org}{jsbml-team@sbml.org}
\section{Supplementary information:} Supplementary data are available at Bioinformatics online.
\end{abstract}

\section{Introduction}

The XML-based Systems Biology Markup Language (SBML, \citealt{Hucka2003}) is
the \emph{de facto} standard file format for the storage and exchange of
biochemical models, and is supported by more than 200
software packages to date (October 2010). Much of this success is due to its
clearly defined specifications and the availability of libSBML \citep{Bornstein2008}, a
portable, robust, and easy-to-use library.

LibSBML provides many methods for manipulating and validating
SBML files through its Application Programming Interface (API).
Originally written in C and C++, libSBML also provides automatically generated 
language bindings for Java\texttrademark, MATLAB\texttrademark, Perl, and
many more. However, the platform independence of Java is compromised
in libSBML due to the fact that the language binding is a
wrapper around the C/C++ core. 
Therefore, many software developers experience difficulties in
the deployment of portable libSBML-based Java applications.
Furthermore, the libSBML API and type hierarchy are not sufficiently intuitive
from a Java programmer's perspective just because they were not
designed directly for Java.
Because of these reasons, among others, several groups in the SBML community
were developing their own native library for SBML. To avoid duplication of
work, some of these groups have mounted an open-source effort to
develop a pure Java library for SBML. Here we present the JSBML
project, whose products are freely available at the web site
\href{http://sbml.org/Software/JSBML}{http://sbml.org/Software/JSBML}.

The aim of the project is to provide an SBML
library that maps all SBML elements to a flexible and extended
type hierarchy. Where possible, JSBML strives to attain
100\,\% compatibility with libSBML's Java API, to ease a switch from
one library to the other. Currently, JSBML supports all constructs
for SBML up to the latest Level 3 Version 1 Core specification. At the moment,
there are no plans to re-implement some of the more complex functions of
libSBML, such as model consistency checking, SBML validation,
and the conversion between different SBML Levels and Versions,
since separate community efforts are expected to make them
available to JSBML via web services.

%\section{Approach}

\begin{methods}
%\section{Methods}
\section{A brief overview of JSBML}

%\begin{itemize}
%\item Support for SBML Levels and Versions (from L1 through L3V1)
%\item Reading/writing of SBML from various sources: Strings, files, URLs, serialized objects, model creation in memory, model manipulation
%\item Communication layer between CellDesigner and program or libSBML and another program
%\item libSBML compatibility
%\item extended, object-oriented SBML type hierarchy based on the SBML specifications with multiple additional abstract data types
%\item support for all SBML constructs (all SBML types, Annotation, incl. MIRIRIAM, Notes in XHTML form -> parser for XHTML strings)
%\item interconvertable math objects: reading and writing in MathML (since SBML Level 2) and C-language like formulas, also writing to \LaTeX{} code; internally only abstract syntax trees
%\item Exceptions for invalid arguments or assignments depending on Level/Version
%\item Over determination check for the model and identification of the variables in algebraic rules
%\item finding elements in the model and in list (filter)
%\item deriving the unit of mathematical expressions
%\item MIRIAM support \citep{Novere2005} and automatic annotation of pre-defined unitdefinitions with corresponding entries of the units ontology.
%\item SBO support and interconvertion between SBO terms and CellDesigner annotation tags.
%\item better adaptation to Java
%\end{itemize}

The main achievement of the JSBML project is its extended type hierarchy that has been designed 
from scratch based on the SBML specifications, but with respect to the naming conventions of
methods and classes in libSBML. For each element that is defined in at least one of the SBML levels 
or versions, JSBML provides a corresponding class that reflects all of its
properties. SBML elements or attributes that are not part of the most 
recent specifications (removed or obsolete) are marked as deprecated. For
elements sharing common properties, JSBML defines a superclass or an interface
for them. For instance, the interface \texttt{NamedSBase} does not correspond to a 
data type in one of the SBML specifications directly, but serves as the
superclass of all those instances of \texttt{SBase} that can be addressed by an identifier and
a name. Similarly, all classes that may contain a mathematical expression implement the 
interface \texttt{MathContainer}. A full overview of this type hierarchy can be found in the supplementary
file. JSBML also supports annotations, including MIRIAM \citep{Novere2005} and SBO \citep{Novere2006b} 
with the help of the BioJava OBO parser \citep{Holland2008}, and notes in XHTML form. Predefined units are automatically
annotated with the correct Unit Ontology terms.
The \texttt{Model} class provides several \texttt{find*} methods to query for SBML elements.
Filters enable to search lists for elements with certain properties. 
All \texttt{ListOf*} elements implement Java's \texttt{List} interface making it 
possible to iterate over its elements and usage of generic types.
Fig.~\ref{fig:JSBML} demonstrates how the hierachically structured content of an SBML file 
can be easily visualized in form of a tree.
\begin{figure*}
\centerline{
  \subfloat[Example source code using JSBML]{\label{lst:JSBMLVisualizer}
    \raisebox{2.7cm}{
      \parbox{.63\textwidth}{
        \begin{minipage}[t][5.7cm][c]{.62\textwidth}
          \lstinputlisting[language=Java]{src/JSBMLvisualizer.java}
        \end{minipage}
      }
    }
  }\hfill
  \subfloat[Example for SBML test case
  26]{\label{fig:JSBMLVisualizer}
%     \raisebox{1.2cm}{
  	\includegraphics[width=.24\textwidth,height=170pt]{img/Case26_Tree_Windows.png}}%}
  	} \caption[A minimal example using JSBML for reading and visualizing an SBML file using JSBML]{
A minimal example using JSBML for reading and visualizing an SBML file. In JSBML, the type \texttt{SBase} extends 
the Java interfaces \texttt{TreeNode}, among others. It allows users
to perform any kind of recursive operations on any instance of \texttt{SBase} such as visualizing it in a \texttt{JTree}.
}
\label{fig:JSBML}
\end{figure*}
Special parsers read mathematical expressions from infix formulas or MathML 
into abstract syntax trees, which are the only internal representation
of formulas in JSBML.
These trees can directly be written in MathML, formula strings, or \LaTeX{} code and 
also be visualized in a \texttt{JTree} since they also implement \texttt{TreeNode}. 
Although JSBML does not implement a fully-featured consistency check for SBML
models, some exceptions are thrown to prevent users to create invalid content.
An overdetermination check for the model has been implemented based on the 
algorithm of \citet{Hopcroft1973}, which also identifies the variables 
in algebraic rules. Furthermore, JSBML can automatically derive the unit of a mathematical expression.
Whenever a property of some \texttt{SBase} is altered, an \texttt{SBaseChangeEvent} is fired
that notifies dedicated listeners. In this way, a graphical user interfaces
could automatically react when any changes happened to the model. With the
help of some modules, JSBML capabilites can be extended and JSBML could be used
as a communication layer between CellDesigner \citep{Funahashi2003} or 
libSBML and any other applications. In this way, JSBML 
facilitates turning an existing application into a plug-in for CellDesigner.
\end{methods}

\section{Implementation}

JSBML is entirely written in the Java\texttrademark{} programming language version 1.5 and does 
not require the installation of any other software besides a Java Virtual Machine. It was successfully
tested under MacOS X, Microsoft Windows XP, Vista, and Windows 7, and openSuSE 11.2 and 
Ubuntu Linux 10.04. JSBML is distributed in form of various JAR files (including or excluding required
third-party libraries) and source code. In addition, a convenient ANT file with several options allows 
users to easily create customized JAR files. Being distributed under the terms of the Lesser GNU
Public License (LGPL), the JSBML library can freely be used even in proprietary software.

%\section{Discussion}


\section{Conclusion}

JSBML is a young, ongoing software project that
provides comprehensive and entirely Java-based data structures
to read, write, and manipulate SBML files. Its layered architecture
allows for the creation of Java web start applications and
CellDesigner plug-ins based on stand-alone programs with very
little effort. One program, SBMLsqueezer \citep{Draeger2008}, has already been
re-implemented and released under version 1.3 using JSBML, a simulator, which is
benchmarked on the SBML test suite will be available soon, and
many other projects are planned.

\section*{Acknowledgement}

\paragraph{Funding\textcolon}

The development of JSBML is funded in part by a grant from the National Institute
of General Medical Sciences (NIGMS, USA), funds from EMBL-EBI (Germany and UK), 
the Federal Ministry of Education and Research (BMBF, Germany) in the Virtual 
Liver Network (grant number 0315756) and the MedSys project Spher4Sys 
(grant number 0315384C), as well as the University of Tuebingen, Germany. 

\paragraph{Conflict of Interest\textcolon} none declared.

%\bibliographystyle{natbib}
%\bibliographystyle{achemnat}
%\bibliographystyle{plainnat}
%\bibliographystyle{abbrv}
%\bibliographystyle{bioinformatics}
%
%\bibliographystyle{plain}
%
%\bibliography{Document}


\bibliographystyle{natbib}
\bibliography{../../literature}
\end{document}
